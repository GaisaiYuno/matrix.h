\documentclass{article}
\usepackage{ctex,amsmath,xparse}
\begin{document}


$A=\begin{bmatrix}
-2 &1 &2\\
1 &0 &0\\
0 &1 &0\\
\end{bmatrix}
$

$A=\begin{vmatrix}
-2-l &1 &2\\
1 &-l &0\\
0 &1 &-l\\
\end{vmatrix}
$=$-2l^2-l^3+l+2$

Factorize it then we have $(l-1)(l+1)(l+2)$

For eigen value $\lambda=1$ ,we have vectors,

$\eta_1=\begin{bmatrix}
1\\
1\\
1\\
\end{bmatrix}
$

For eigen value $\lambda=-1$ ,we have vectors,

$\eta_1=\begin{bmatrix}
1\\
-1\\
1\\
\end{bmatrix}
$

For eigen value $\lambda=-2$ ,we have vectors,

$\eta_1=\begin{bmatrix}
4\\
-2\\
1\\
\end{bmatrix}
$

Combine the vectors together then we get, 

$P=\begin{bmatrix}
1 &1 &4\\
1 &-1 &-2\\
1 &1 &1\\
\end{bmatrix}
$

$\Lambda=PAP^{-1}=\begin{bmatrix}
1 &0 &0\\
0 &-1 &0\\
0 &0 &-2\\
\end{bmatrix}
$

So $A^n=P^{-1}\Lambda^n P$

For calculating $A^n$, assume $\Lambda^n=[a,0,0;0,b,0;0,0,c]$, then we have:

$\frac{1}{6}a-\frac{1}{2}b+\frac{4}{3}c$

Then, substitute $a=(1)^{n-1}$ and $b=(-1)^{n-1}$ and $c=(-2)^{n-1}$, we have the final answer

\end{document}


